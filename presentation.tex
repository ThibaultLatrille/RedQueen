\documentclass[10pt]{beamer}

\usepackage[T1]{fontenc}
\usepackage[utf8]{inputenc}
\usepackage{graphicx}
\usepackage{hyperref}
\usepackage{lmodern}
\usepackage{listings}
\usepackage{amssymb} 
\usepackage{xcolor}
\usepackage{tikz}
\usepackage{bm}
\usebackgroundtemplate{\tikz\node[opacity=0]{};}
\usepackage{adjustbox}
\newcommand{\specialcell}[2][c]{%
	\begin{tabular}[#1]{@{}c@{}}#2\end{tabular}}

%gets rid of bottom navigation overlines
\setbeamertemplate{footline}[frame number]{}

%gets rid of bottom navigation symbols
\setbeamertemplate{navigation symbols}{}

%gets rid of footer
%will override 'frame number' instruction above
%comment out to revert to previous/default definitions
\setbeamertemplate{footline}{}

\sloppy 

\definecolor{RED}{HTML}{EB6231}
\definecolor{YELLOW}{HTML}{E29D26}
\definecolor{BLUE}{HTML}{5D80B4}
\definecolor{LIGHTGREEN}{HTML}{6ABD9B}
\definecolor{GREEN}{HTML}{8FB03E}
\definecolor{PURPLE}{HTML}{BE1E2D}
\definecolor{BROWN}{HTML}{A97C50}
\definecolor{PINK}{HTML}{DA1C5C}

\newcommand{\avg}[1]{\left< #1 \right>} % for average
\newcommand{\Ne}{N_\mathrm{e}}
\newcommand{\Rp}{r}
\newcommand{\R}{\bm{\textcolor{GREEN}{R}}}
\newcommand{\Rt}{\bm{\textcolor{GREEN}{\R(}}t\bm{\textcolor{GREEN}{)}}}
\newcommand{\x}{\bm{\textcolor{BLUE}{x(}\textcolor{YELLOW}{\Rp}\textcolor{BLUE}{)}}}
\newcommand{\ratec}{\bm{\textcolor{LIGHTGREEN}{c}}}
\newcommand{\xinitial}{\bm{\textcolor{BLUE}{x_{initial}}}}
\newcommand{\re}{\bm{\textcolor{YELLOW}{\Rp}}}
\newcommand{\xp}{\bm{\textcolor{BLUE}{x(}}t\bm{\textcolor{BLUE}{)}}}
\newcommand{\lp}{\bm{\textcolor{YELLOW}{\Rp(}}t\bm{\textcolor{YELLOW}{)}}}
\newcommand{\xpi}{\bm{\textcolor{BLUE}{x_i(}}t\bm{\textcolor{BLUE}{)}}}
\newcommand{\xpj}{\bm{\textcolor{BLUE}{x_j(}}t\bm{\textcolor{BLUE}{)}}}
\newcommand{\lpi}{\bm{\textcolor{YELLOW}{\Rp_i(}}t\bm{\textcolor{YELLOW}{)}}}
\newcommand{\lpic}{\bm{\textcolor{YELLOW}{\Rp_{i,\ratec}(}}t\bm{\textcolor{YELLOW}{)}}}
\newcommand{\lpj}{\bm{\textcolor{YELLOW}{\Rp_j(}}t\bm{\textcolor{YELLOW}{)}}}
\newcommand{\Rmin}{\bm{\textcolor{RED}{R_{\infty}}}}
\newcommand{\D}{\bm{\textcolor{PURPLE}{D}}}
\newcommand{\taup}{\bm{\textcolor{BROWN}{\tau}}}
\newcommand{\dd}{\mathrm{d}}



\begin{document}
	
	\begin{frame}
		\vspace{1cm}
		\begin{enumerate}
			\item The locus of PRDM9 mutates at constant rate $u$ per generation.\\
			\item Each mutation produces a new functional PRDM9 allele.\\
			\item The number of targets is the same for each allele.\\
			\item There is no overlap between the targets of distinct PRDM9 alleles.\\
			\item $K(t)$ denotes the number of PRDM9 alleles in the population.\\
			\item $n_i(t)$ is the number of copies of the $i^{th}$ allele in the population.\\
			\item $\xpi = n_i(t) / 2 \Ne$ is the frequency of allele $i$ at time $t$.\\
		\end{enumerate}
		\vspace{15pt}
		\begin{equation*}
		\end{equation*}
	\end{frame}
	
	\begin{frame}
		\vspace{1cm}
		\begin{enumerate}
			\item The recombination activity induced by an allele is maximal at the birth of this allele .\\
			\item Erosion is modelled implicitly, by tracking over time the fraction of active targets associated with each allele, $\lpi$.\\
			\item $v$ is the mutation rate at the target sites.\\
			\item The rate of inactivating mutations per target at the level of the population is $2 \Ne v$
			\item $g$ is the rate of conversion of active targets by the inactive mutant in an heterozygous individual.\\
			\item Under strong dBGC, the fixation probability of inactive mutant equal to $2g \xpi$ for $i^{th}$ allele of PRDM9.
		\end{enumerate}
		\vspace{10pt}
		Altogether, the activity induced by allele $i$ decays as:
		\begin{equation*}
		\dfrac{\dd \lpi}{\dd t} = - \rho \xpi \lpi \text{, where } \rho = 4 \Ne vg 
		\end{equation*}
	\end{frame}
	
	\begin{frame}
		\vspace{1cm}
		\begin{enumerate}
			\item The fitness of an individual with genotype $(i, j)$ is: $$\omega_{i,j}(t)=f\left( \dfrac{\lpi+ \lpj}{2}\right)$$\\
			\item $f$ is assumed to be an increasing function, $f(x)=x^{\alpha}$.\\
			\item The average fitness induced by allele $i$ over the population is then $$ \omega_i (t) = \sum_{j=1}^{K_t} \xpj \omega_{i,j}(t) $$
			\item The mean fitness over the population is $$ \overline{\omega(t)} = \sum_{i=1}^{K_t} \xpi \omega_{i}(t) $$
			\item The probability for the $i^{th}$ allele to be picked up at the next generation is: $$ p_i(t+1) = \xpi \dfrac{\omega_i (t)}{\overline{\omega(t)}} $$
		\end{enumerate}
	\end{frame}
	
	\begin{frame}
		\begin{enumerate}
			\item $\lp$ is the activity of targets for the current PRDM9 allele. $$ 
			\dfrac{\dd \lp}{\dd t} = - \rho \lp \Rightarrow \lp = e^{- \rho t}\text{, where } \rho = 4 \Ne v g$$
			\item $\taup$ the mean time between two successive invasions.
			$$\Rightarrow \R = \dfrac{1}{\taup} \int_{0}^{\taup} \lp \dd t = \dfrac{1}{\taup} \int_{0}^{\taup} e^{- \rho t} \dd t = \dfrac{1 - e^{- \rho \taup }}{\rho \taup } $$
			\item $s_0$ is the selection coefficient experienced by a new allele $$ s_0 \simeq \dfrac{f'(\lp)}{f(\lp)}\dfrac{1-\lp}{2} \simeq \dfrac{f'(\R)}{f(\R)}\dfrac{1-\R}{2} $$
			\item $\taup$ is also the inverse of the invasion rate:$$
			\taup = \dfrac{1}{\mu s_0 } \simeq \dfrac{1}{\mu} \dfrac{f(\R)}{f'(\R)}\dfrac{2}{1-\R}\text{, where } \mu = 4 \Ne u  $$
			\item Altogether,
			$$ \R = g\left( \R, \dfrac{\rho}{\mu}\right) = g\left( \R, \dfrac{vg}{u}\right)  $$
		\end{enumerate}
	\end{frame}
	
	\begin{frame}
		\vspace{1cm}
		\begin{enumerate}
			\item $\xpi$ is the frequency of the $i^{th}$ PRDM9 allele.\\
			\item $\lpi$ is the target's activity associated to the $i^{th}$ PRDM9 allele.\\
			\item Strong selection (no drift).
			$$
			\left\{
			\begin{aligned}
			\dfrac{\dd \xpi}{\dd t} &= \dfrac{f'(\Rt)}{ 2 f(\Rt) } \left( \lpi  - \Rt  \right) \xpi\\
			\dfrac{\dd \lpi}{\dd t} &= - \rho \xpi \lpi\text{, where } \rho = 4 \Ne v g \\
			\Rt &= \sum_{i } \xpi \lpi
			\end{aligned}
			\right. 
			$$
			\item $\Rt$ approximated as a constant parameter $\R$ (mean-field):
			$$
			\left\{
			\begin{aligned}
			\dfrac{\dd \xp }{\dd t} &=  \dfrac{f'(\R)}{ 2 f(\R) } \left( \lp  - \R \right) \xp \\
			\dfrac{\dd \lp}{\dd t} &= 
			- \rho \xp \lp \\
			\end{aligned}
			\right.
			$$
		\end{enumerate}
	\end{frame}
	
	\begin{frame}
		\begin{equation*}
		\hspace*{-0.5cm}
		\left\{
		\begin{aligned}
		\dfrac{\dd \xp }{\dd t} &= \dfrac{f'(\R)}{2 f(\R) } \left( \lp - \R \right) \xp\\
		\dfrac{\dd \lp }{\dd t} &= 
		- \rho  \xp  \lp \\
		\end{aligned}
		\right.
		\Rightarrow  \left\{
		\begin{aligned}
		\x &= \dfrac{f'(\R)}{2 \rho f(\R) }  \left[1 - \re + \R \mathrm{ln}(\re)\right] + \xinitial  \\
		0 &= 1 - \Rmin + \R \mathrm{ln}(\Rmin) \\
		\end{aligned}
		\right.
		\end{equation*}
		$$ \sum_i \xpi = 1 \Leftrightarrow \taup = \int_{0}^{\infty} \xp \dd t $$
	\end{frame}
	
	\begin{frame}
		\begin{enumerate}
			\item We have a relation between $\Rmin$ and $\R$:
			$$ 0 = 1 - \Rmin + \R \mathrm{ln}(\Rmin) $$ \\
			\item From the tilling argument, we also have: 
			$$\taup = \int_{0}^{\infty} \xp \dd t= \dfrac{1 - \Rmin }{\rho \R} \Leftrightarrow \R = \dfrac{1 - e^{- \rho \taup }}{\rho \taup }\text{, where } \rho = 4 \Ne v g$$
			\item As in succesion regime, $\taup$ is also the inverse of the invasion rate : 
			$$\taup \simeq \dfrac{1}{\mu} \dfrac{f(\R)}{f'(\R)}\dfrac{2}{1-\R}\text{, where } \mu = 4 \Ne u $$
			\item Altogether, we get the exact same equation as in succession regime: 
			$$\R = g\left( \R, \dfrac{\rho}{\mu}\right) = g\left( \R, \dfrac{vg}{u}\right)$$
		\end{enumerate}
	\end{frame}
	
	\begin{frame}
		\begin{enumerate}
			\item Recombination rates across hot spots	vary according to a gamma distribution of mean $1$ and shape parameter $a$: 
			$$  p(\ratec) = \dfrac{b^a}{\Gamma (a)} \ratec^{a-1} e^{-a \ratec} $$
			\item The rate of erosion for the fraction of hot spots recombining at rate $\ratec$ decays at a rate proportional to $\ratec$: 
			$$ \dfrac{\dd \lpic}{\dd t} = - \rho \xpi \ratec \lpic $$
			\item The fraction of active targets in the population is then:
			$$ \R = \dfrac{1}{\rho \taup} \dfrac{a}{(a-1)}\left[1-\left( \dfrac{a}{a + \rho \taup} \right)^{a-1}\right]$$ 
		\end{enumerate}
	\end{frame}
	
	\begin{frame}
		$$ \R = \avg{\sum_{i} \xpi \lpi } $$ 
		$$ \D = \avg{\dfrac{1}{\sum_{i} \xpi^2} } $$
		$$\dfrac{vg}{u} \ll 1 \Rightarrow \D \simeq 24 \Ne u $$ 
		$$\dfrac{vg}{u} \ll 1 \Rightarrow 1 - \R \propto \sqrt{\dfrac{vg}{u}}$$
	\end{frame}
	
	\begin{frame}
		\begin{enumerate}
			\item $\D \simeq 7$, estimated between $5$ to $10$.
			\item $\R \simeq 0.5 $, since the major allele eroded 50\% of it's targets.
			\item $S=4\Ne s_0 \gg 1$, suggested by the presence of strong
			positive selection acting on the Zn-finger array of PRDM9.
			\item $\Ne \simeq 10^5$, ranging from $\Ne = 5.10^4$ to $\Ne = 5.10^5$.
			\item $v \simeq 10^{-7}$, assuming a point mutation rate of $10^{-8}$ and $10$ inactivating mutations per target.
			\item $\Ne$ and $v$ are known. $3$ parameters left to estimate: $u$, $g$ and $\alpha$. 
		\end{enumerate}
		\vspace{-10pt}
		\begin{table}[ht]
			\centering
			\begin{adjustbox}{width = 1\textwidth}
				\small\begin{tabular}{|c|c|c|c|c|c|c|c|}
					\hline
					\specialcell{Mutation rate of \\ PRDM9 ($u$)} & \specialcell{Erosion rate of \\ targets ($vg$)} & \specialcell{Fitness \\ parameter ($\alpha$)} & $\epsilon=\dfrac{vg}{u}$ & \specialcell{Mean fraction of \\ active targets ($\R$)} & \specialcell{Diversity at \\ PRDM9 locus ($\D$)} & \specialcell{Scaled selection \\ coefficient ($S$)} & \specialcell{Turn-over \\ time ($T$)}\\
					\hline
					$3 \times 10^{-6}$ & $3 \times 10^{-10}$ & $1 \times 10^{-4}$ & $1 \times 10^{-4}$ & $0.6$ & $9.9$ & $26$ & $6.4 \times 10^{4}$\\
					$3 \times 10^{-6}$ & $3 \times 10^{-11}$ & $1 \times 10^{-4}$ & $1 \times 10^{-5}$ & $0.82$ & $8.2$ & $8.6$ & $1.6 \times 10^{5}$\\
					$3 \times 10^{-7}$ & $3 \times 10^{-11}$ & $1 \times 10^{-4}$ & $1 \times 10^{-4}$ & $0.6$ & $1$ & $26$ & $6.5 \times 10^{4}$\\
					$3 \times 10^{-6}$ & $3 \times 10^{-11}$ & $1 \times 10^{-5}$ & $1 \times 10^{-5}$ & $0.6$ & $9.9$ & $2.6$ & $6.4 \times 10^{5}$\\
					\hline
				\end{tabular}
			\end{adjustbox}
			\caption{Fitness function is a power law, $f(x)=x^{\alpha}$}
		\end{table}
		\vspace{-10pt}
		\begin{enumerate}
			\item[7.] $u \simeq 3.10^{-6}$, $g \simeq 3.10^{-3}$ and $\alpha \simeq 10^{-4}$.
		\end{enumerate}
	\end{frame}
\end{document}


