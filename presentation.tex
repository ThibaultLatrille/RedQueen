\documentclass[10pt]{beamer}
 
\usepackage[T1]{fontenc}
\usepackage[utf8]{inputenc}
\usepackage{graphicx}
\usepackage{hyperref}
\usepackage{lmodern}
\usepackage{listings}
\usepackage{amssymb} 
\usepackage{xcolor}
\usepackage{tikz}
\usepackage{bm}
\usebackgroundtemplate{\tikz\node[opacity=0]{};}


%gets rid of bottom navigation overlines
\setbeamertemplate{footline}[frame number]{}

%gets rid of bottom navigation symbols
\setbeamertemplate{navigation symbols}{}

%gets rid of footer
%will override 'frame number' instruction above
%comment out to revert to previous/default definitions
\setbeamertemplate{footline}{}

\author{Latrille Thibault, Laurent Duret, Nicolas Lartillot}
\title{The red queen in the kingdom of recombination.}  
\institute{Laboratoire de Biométrie et Biologie Évolutive (LBBE), UMR CNRS 5558, Lyon}

\sloppy 

\definecolor{RED}{HTML}{EB6231}
\definecolor{YELLOW}{HTML}{E29D26}
\definecolor{BLUE}{HTML}{5D80B4}
\definecolor{LIGHTGREEN}{HTML}{6ABD9B}
\definecolor{GREEN}{HTML}{8FB03E}

\newcommand{\Ne}{N_\mathrm{e}}
\newcommand{\Lb}{\bm{\textcolor{GREEN}{\overline{L}}}}
\newcommand{\Linf}{\bm{\textcolor{RED}{L_{\infty}}}}
\newcommand{\xp}{\bm{\textcolor{BLUE}{x}}}
\newcommand{\lp}{\bm{\textcolor{YELLOW}{l}}}

\begin{document}

\frame{\titlepage} 

\begin{frame}
\vspace{2cm}
	$\bullet$ $ K(t)$ is the number of variants in the population.
	
	$\bullet$ $\forall i \in \{ 1, \, \dots, \, K(t) \}$:
		
	$\hspace{0.3cm}  \bullet$ $x_i(t)$ is the frequency of the $i^{th}$ PRDM9 variant.\\
	
	$\hspace{0.3cm} \bullet$ $l_i(t)$ is the proportion of active hotspots  associated to the $i^{th}$ PRDM9 variant.\\
	
	$\bullet$ $N_e$ is the population size.\\
	
	$\bullet$ $v$ is the mutation rate at the hotspot.\\
	
	$\bullet$ $r_0$ is the recombination rate.\\
		\begin{enumerate}
	\item Substitution from active to inactive at a rate $ 2 N_e v  l_i(t) $ \\
	\item Selection at a rate $2 r_0 x_i(t) $
	\end{enumerate}
\[
      \begin{aligned}
        \dfrac{\mathrm{d}l_i(t)}{\mathrm{d}t} &= 
        - 4 N_e v r_0 x_i(t) l_i(t) \\
      \end{aligned}
\]
\end{frame}

\begin{frame}
\vspace{2cm}
	$\bullet$ $ K(t)$ is the number of variants in the population.
	
	$\bullet$ $\forall i \in \{ 1, \, \dots, \, K(t) \}$:
		
	$\hspace{0.3cm}  \bullet$ $x_i(t)$ is the frequency of the $i^{th}$ PRDM9 variant.\\
	
	$\hspace{0.3cm} \bullet$ $l_i(t)$ is the proportion of active hotspots  associated to the $i^{th}$ PRDM9 variant.\\
	
	$\bullet$ $N_e$ is the population size.\\
	
	$\bullet$ $u$ is the mutation rate of PRDM9.\\

	$\bullet$ $\overline{\omega_i}=\sum_j x_j(t) f \left( \tfrac{l_i(t) + l_j(t)}{2} \right)$ is the fitness of the $i^{th}$ PRDM9 variant.\\
	
	$\bullet$ $\overline{\omega}=\sum_{i} x_i(t) \overline{\omega_i}$ is the mean fitness in the population.\\
		\begin{enumerate}
	\item New variant from a Poisson distribution at a rate $ 2 N_e u $ per generation. \\
	\item The probability of drawing variant $i$ in the new generation is $ \tfrac{x_i(t) \overline{\omega_i}}{\overline{\omega}} $
		\end{enumerate}
\end{frame}

\begin{frame}
\vspace{2cm}
	Let $x_i$ be the frequency of the $i^{th}$ PRDM9 variant.\\

    \[ K_e = \dfrac{1}{\displaystyle \sum_i x_i^2}   \]
\end{frame}

\begin{frame}
\vspace{2cm}
	$ N_e $ is the population size. \\ 
	$ u $ is the mutation rate of PRDM9. \\
	\begin{enumerate}
		\item $ N_e u \ll 1 \Rightarrow  K_e \simeq 1$ (single allele succession)
		\item $ N_e u \gg 1 \Rightarrow  K_e > 1 $ (polymorphism)
		
		\item $K_e \nearrow$ with $N_e$ and $u$.
		
		\item $K_e \rightsquigarrow$ with the mutation ($v$) and recombination rate at the hotspots ($r_o$)
		
		\item $K_e \rightsquigarrow$ with the fitness function.
	\end{enumerate}
\end{frame}


\begin{frame}
\vspace{2cm}
	Let $x_i$ be the frequency of the $i^{th}$ PRDM9 variant.\\
	Let $l_i$ be the fraction of hot hotspots (still not eroded) associated with the $i^{th}$ PRDM9 variant.\\

    \[ \Lb =  \sum_i x_i l_i   \]
\end{frame}

\begin{frame}
\vspace{2cm}
	\begin{enumerate}
			
		\item $\Lb \rightsquigarrow$ with the population size ($N_e$).
				
		\item $\Lb \nearrow$ with the mutation rate of PRDM9. ($u$).
		
		\item $\Lb \searrow$ with the mutation ($v$) and recombination rate at the hotspots ($r_o$).

		\item 	The fitness function can be linearised around the mean erosion $\Lb$.
	\end{enumerate}
\end{frame}

\begin{frame}
\vspace{2cm}
	$\tau$ is the number of generations needed to replace all PRDM9 variants. \\
	It is such that the homozygosity between $x_i(t)$ and $x_i(t+\tau)$ vanishes.
\end{frame}

\begin{frame}
\vspace{2cm}
	\begin{enumerate}
		\item $\tau \searrow$ and $\rightsquigarrow$ with the population size ($N_e$).
			
		\item $\tau \searrow$ and $\nearrow$ with the mutation rate of PRDM9. ($u$).
		
		\item $\tau \searrow$ with the mutation ($v$) and recombination rate at the hotspots ($r_o$).
	\end{enumerate}
\end{frame}

\begin{frame}
\vspace{2cm}
	\begin{enumerate}
	\item $\forall i \in \{ 1, \, \dots, \, K \} $, $K$ is the number of variants in the population.
		
	\item $x_i$ is the frequency of the $i^{th}$ PRDM9 variant.\\
	
	\item $l_i$ is the proportion of hot hotspots associated to the $i^{th}$ PRDM9 variant.\\
		
	\item Assume there is no drift. 
	
	\item Linearise the fitness function around the mean erosion $\Lb = \sum_i l_i x_i$.
	\end{enumerate}
\vspace{15pt}
\[
  \left\{
      \begin{aligned}
          \dfrac{\mathrm{d}x_i}{\mathrm{d}t} &= \dfrac{f'(\Lb)}{2 f(\Lb)} \left( l_i - \Lb \right) x_i \\
        \dfrac{\mathrm{d}l_i}{\mathrm{d}t} &= 
        - \rho x_i l_i \\
      \end{aligned}
    \right.
\]
\end{frame}

\begin{frame}
\vspace{1.5cm}
	\begin{enumerate}
		
	\item $\xp(t)$ is the frequency of PRDM9.\\
	
	\item $\lp(t)$ is the proportion of active hotspots associated to PRDM9.\\
		
	\item Ignore drift. 
			
	\item Linearise the fitness function.
	
	\item Approximate $\Lb$ as a constant parameter.


	\end{enumerate}
\vspace{15pt}
	\[
  \left\{
      \begin{aligned}
          \dfrac{\mathrm{d}\xp(t)}{\mathrm{d}t} &= \dfrac{f'(\Lb)}{2 f(\Lb)} \left( \lp(t) - \Lb \right) \xp(t) \\
        \dfrac{\mathrm{d}\lp(t)}{\mathrm{d}t} &= 
        - \rho \xp(t) \lp(t) \\
      \end{aligned}
    \right.
\]
\end{frame}


\begin{frame}
\vspace{2cm}
	\begin{enumerate}
		
	\item $\Lb$ is the mean activity of the hotspots.\\
	
	\item $\Linf$ is the minimum activity of the hotspots.\\
		
	\item $f$ is the fitness function.
	\end{enumerate}
\vspace{15pt}
\[
          \xp(\lp) =\dfrac{f'(\Lb)}{2 \rho f(\Lb)} (1-\lp + \Lb \mathrm{log}(\lp)) + \xp_{\mathrm{initial}} 
\]
\vspace{5pt}
\[
         \Rightarrow  \xp(\Linf)  \simeq  1-\Linf + \Lb \mathrm{log}(\Linf) = 0 
\]
\end{frame}


\begin{frame}
\vspace{1.5cm}
	\begin{enumerate}
		
	\item $\xp(t)$ is the frequency of PRDM9.\\
	
	\item $\lp(t)$ is the proportion of active hotspots associated to PRDM9.\\
		
	\item $\Lb$ is the mean activity of the hotspots.\\
	
	\item $\Linf$ is the minimum activity of the hotspots.\\
	\end{enumerate}
\vspace{5pt}
\[
  \left\{
      \begin{aligned}
          \xp(\lp) &= \dfrac{f'(\Lb)}{2 \rho f(\Lb)} (1-\lp + \Lb \mathrm{log}(\lp)) + \xp_{\mathrm{initial}}  \\
         0 &= 1-\Linf + \Lb \mathrm{log}(\Linf) \\
        \dfrac{v r_o}{2 u} &= \dfrac{f'(\Lb)}{f(\Lb)} \dfrac{(1- \Lb)(1- \Linf)}{\Lb}  \\
      \end{aligned}
    \right.
\]
\vspace{5pt}
\Large
\[
    \Rightarrow 
    g(\Lb) = \dfrac{v r_o}{2 u}
\]
\end{frame}

\begin{frame}
\vspace{2cm}
\[
  K_e(\Lb) \simeq 
  \dfrac{4 \rho f(\Lb)}{f'(\Lb)\left[ 1 + \Linf - 2 \Lb  \right]}
\]
\end{frame}

\begin{frame}
\vspace{2cm}
\[
  \tau (\Lb) \sim \dfrac{2 f'(\Lb)}{f(\Lb)[\Lb-1 + \Lb \mathrm{log}(\Lb)]}
\]
\end{frame}

\begin{frame}
\large
\vspace{3.5cm}
\begin{equation}
N_e u \gg 1 \Rightarrow  K_e > 1 
\end{equation}\\
\begin{equation}
1-\Linf + \Lb \mathrm{log}(\Linf) = 0
\end{equation}\\
\begin{equation}
\dfrac{f'(\Lb)}{f(\Lb)} \dfrac{(1- \Lb)(1- \Linf)}{\Lb} = \dfrac{v r_o}{2 u}
\end{equation}
\end{frame}

\end{document}


