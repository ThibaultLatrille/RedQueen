\documentclass{article}

\usepackage[T1]{fontenc}
\usepackage[utf8]{inputenc}
%\usepackage[french]{babel}
\usepackage{graphicx}
\usepackage{hyperref}
\usepackage{lmodern}
\usepackage{amsmath}
\usepackage{amsthm}
\usepackage{listings}
\usepackage{enumerate}
\usepackage{amssymb}
\usepackage{amsfonts}
\usepackage{float}

\usepackage[a4paper]{geometry}

\author{Latrille Thibault, Laurent Duret, Nicolas Lartillot}
\title{The red queen dynamic in the kingdom of recombination.}  

\sloppy 

\begin{document}

\maketitle 

\newcommand{\Ne}{N_\mathrm{e}}

%\tableofcontents             

\section{Abstract}

In humans and many other species, recombination events cluster into narrow hotspots within the genome. Given the vital role recombination plays in meiosis, we might expect that the positions of these hotspots would be tightly conserved over evolutionary time. However, there is now strong evidence that hotspots of meiotic recombination in humans are transient features of the genome. For example, hotspot locations are not shared between human and chimpanzee. Biased gene conversion in favor of alleles that locally disrupt hotspots is a possible explanation of the short lifespan of hotspots.

Remarkably, Prdm9 has been proposed to be a key determinant of the positioning of recombination hotspots during meiosis, and the most rapidly evolving gene in human. Prdm9 genes often exhibit substantial variation in their numbers of encoded zincfingers, not only between closely related species but also among individuals of a species.

Here, we propose a population genetic model which exhibits hotspots transience while reflecting the PRDM9 features, resulting in a intragenomic red queen dynamic. Our model account for empirical observations regarding the molecular mechanisms of recombination hotspots and the nonrandom targeting of the recombination by PRDM9. We further investigate and compare to known data the diversity of PRDM9, the hotspots turnover and the genome wide disruption of hotspots.

\bibliographystyle{plain}

\end{document}





