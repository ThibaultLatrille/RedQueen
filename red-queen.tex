\documentclass{article}

\usepackage[T1]{fontenc}
\usepackage[utf8]{inputenc}
\usepackage{graphicx}
\usepackage{hyperref}
\usepackage{lmodern}
\usepackage{amssymb,amsfonts,amsmath,amsthm}
\usepackage{listings}
\usepackage{enumerate}
\usepackage{amssymb}
\usepackage{amsfonts}
\usepackage{multicol}
\usepackage{float}

\usepackage[margin=60pt]{geometry}

\author{Thibault Latrille, Laurent Duret, Nicolas Lartillot}
\title{The red queen dynamic in the kingdom of recombination.}  

\sloppy 

\begin{document}

\maketitle 

\newcommand{\avg}[1]{\left< #1 \right>} % for average
\newcommand{\Ne}{N_\mathrm{e}}
\newcommand{\Rp}{\theta}
\newcommand{\R}{R}
\newcommand{\Rt}{\R_{t}}
\newcommand{\D}{D}
\newcommand{\Dt}{\D_{t}}
\newcommand{\V}{V}
\newcommand{\Vt}{\V_{t}}
\newcommand{\T}{T}
\newcommand{\Rmin}{{\R_{\infty}}}
\newcommand{\dd}{\mathrm{d}}


\newcommand{\Lr}{L}
\newcommand{\Lmin}{\Lr_{\infty}}
\newcommand{\Lmina}{\alpha}
\newcommand{\Lminb}{\beta}
\newcommand{\Lra}{\gamma}
\newcommand{\Lrb}{\delta}
%\tableofcontents             

\section*{Abstract}
In humans and many other species, recombination events cluster in narrow hotspots distributed across the genome, whose location is determined by the Zing-finger protein PRDM9. Surprisingly, hotspots are not shared between human and chimpanzee, suggesting that hotspots are short-lived. To explain this fast evolutionary dynamics of recombination landscapes, an intra-genomic Red-Queen model, based on the interplay between two antagonistic forces, has been proposed. On the one hand, biased gene conversion, mediated by double-strand breaks, results in a rapid extinction of hotspots in the population. On the other hand, the resulting genome-wide depletion of recombination induces strong positive selection favoring new Prdm9 alleles recognizing new sequence motifs across the genome, thereby restoring normal levels of recombination. Thus far, however, this Red-Queen scenario has not been formalized as a quantitative model.

Here, we propose a detailed population-genetic model of the Red-Queen dynamic of recombination. This model was implemented as a Wright-Fisher simulator, allowing exploration of the behaviour of the model (in terms of the implied mean equilibrium recombination rate, diversity at the PRDM9 locus, or turnover rate) as a function of the parameters (effective population size, mutation rate). In a second step, analytical results, based on self-consistent mean-field approximations, were derived. These analytical results reproduce the scaling relations observed in the simulations, offering key insights about the detailed population-genetic mechanisms of the Red-Queen model. These insights and scaling relations can now be tested against empirical data currently available in mammals.

\section*{Introduction}
PRDM9 is a meiosis-specific histone methyltransferase with a tandem-repeat zinc finger (ZnF) domain encoded by a mini-satellite-like sequence. The ZnF domain is polymorphic in repeat number and type, and appears to be directly responsible for activating recombination hotspots by binding to hotspot-associated sequence motifs in both humans and mice.

Hotspots evolve rapidly, as shown by the totally different fine-scale recombination landscapes of humans and chimpanzees.
Turnover might be driven by the tendency of hotspots to self-destruct through the systematic over-transmission of variants within hotspots that down-regulate recombination initiation, leading to hotspot depletion and consequent 
selection in favour of PRDM9 variants that activate new sets of hotspots.

We are interested in a mathematical modelling of the turnover of PRDM9 variants and the associated recombination hotspots.

\section*{Materials and methods}

\subsection*{Population genetic model}

\subsection*{PRDM9 mutation}

The population is composed of $\Ne$ diploid individuals, kept constant over time.

The locus PRDM9 mutates at constant rate $u$ per generation per locus and each new variant binds a new sets of sequence targets. Thus $u$ can be understood as a functional mutation rate, with each mutation producing a new PRDM9 variant with new hotspots.

At each generation, $K_{t}$ is the number of PRDM9 variants in the population. $\forall i \in \{ 1, \, \dots, \, K_{t} \}$, $n_{i,t}$ is the number of $i^{th}$ PRDM9 variant in the population. Consequently, $x_{i,t} = n_{i,t} / 2 \Ne$ is the frequency of the $i^{th}$ PRDM9 variant.

\subsection*{Erosion of the hotspots}

PRDM9 activates it's targets in \textit{trans}. The number of new mutation occurring in the population at the targets is $ \Ne v$, each of them are driven by a selection coefficient $ r _0  x_{i,t}$, where $x_{i,t}$ is the probability of activation by the variant $i$.
Since the strength of activity is proportional to the targets being not eroded, we model the activity as an exponential decay.

We consider that PRDM9 is binding it's target in *trans*, meaning there is no linkage between the locus of PRDM9 and any of the target. The sequence targets of PRDM9 mutates at constant rate $v$ per generation. The sequence targets recombine at constant rate $r_0$ per generation. The recombination at the sequence target is favoured by the binding of PRDM9.

For each variant, $\Rp _{i,t}$ is the overall activity of the sequence targets, with $\Rp _{i,t}=1$ meaning no activity and $\Rp _{i,t}=0$ meaning totally eroded. $\overline{\omega_{i,t}}$ is fitness of this variant. At the population level, $\overline{\omega}=\sum_{i} x_{i,t} \overline{\omega_{i,t}}$ is the mean fitness.

\subsection*{Selection of PRDM9}

We propose $\overline{\omega_{i,t}}=\sum_{j \in K_{t}} x_{j,t} f \left( \dfrac{\Rp _{i,t} + \Rp _{j,t}}{2} \right)$, for any function $f\colon [0,1] \rightarrow \mathbb{R}^+$. Meaning the fitness of variant $i$ is the sum over the probabilities that the second variant is $j$ (diploid individuals) time a function of the mean activity between variant $i$ and $j$. Thus we assume the variants interact linearly.

\subsection*{Overall simulation cycle} 

For each new generation, the simulation is decomposed in three steps, computed in the following order: \\

1. Mutation and creation of new variants of PRDM9 

Per locus, PRDM9 mutates at constant rate $u$, and we have $2 \Ne$ loci in the population. Thus the number of new variants is Poisson distributed with mean $2 \Ne u$. The new variants are introduced in the population at a frequency $1 / 2 \Ne$

\begin{equation}
  K_{t+1} - K_{t} \sim \operatorname{Pois} \left(2 \Ne u \right)
\end{equation}

2. Erosion of the recombination hotspots 

PRDM9 activates it's targets in \textit{trans}. The number of new mutation occurring in the population at the targets is $ \Ne v$, each of them are driven by a selection coefficient $ r _0  x_{i,t}$, where $x_{i,t}$ is the probability of activation by the variant $i$.
Since the strength of activity is proportional to the targets being not eroded, we model the activity as an exponential decay.

\begin{align}
 \Rp _{i,t+1} &=  \Rp _{i,t}\operatorname{exp} \left( - \Ne v
 r _0  x_{i,t} \right), \;
 \forall i \in \{ 1, \, \dots, \, K \} \\ \\
\end{align}

3. Drift and selection

The new generation of $2 \Ne$ PRDM9 alleles is drawn from a multinomial distribution, generating a drift. The probability of drawing variant $i$ is equal to it's frequency $x_{i,t}$ time it's relative fitness  $\dfrac{\overline{\omega_{i,t}}}{\overline{\omega}}$. The probabilities sum to $1$ by definition of $\overline{\omega}$.

\begin{align}
  & \left(
  n_{1, t+1}, \,
  \dots, \,
  n_{i,t+1}, \,
  \dots, \,
  n_{K_{t+1}}
  \right)\\ \\
  &  \, \sim \operatorname{Multinomial} \left(2 \Ne, \,
  \dfrac{x_{1,t}  \overline{\omega_{1,t}}}{\overline{\omega}}, \,
  \dots, \,
  \dfrac{x_{i,t}  \overline{\omega_{i,t}}}{\overline{\omega}}, \,
  \dots, \,
  \dfrac{x_{K_{t}} \overline{\omega_{K_{t}}}}{\overline{\omega}}  \right)
\end{align}

The model was implemented numerically by Monte Carlo simulations in Python, the code is hosted at \url{https://github.com/ThibaultLatrille/RedQueen}.

\subsection*{Summary statistics} 

To explore the behaviour of the Wright-Fisher model as a function of the parameters, one need to have summary statistics that gives insight on the dynamic of the red-queen. We focus on four measures, 
the diversity at PRDM9 locus, the mean recombination rate, 
Summary statistics are computed when the simulation reaches the limit cycle (attractor) of the red-queen. In out simulations are taken as the mean over many cycles (100 cycles). 

1. Diversity of PRDM9

Diversity can be defined in several ways; for example the richness $K_{t}$, defined above as the number of variants, is a measure of PRDM9 diversity.
One shortcoming for $K_{t}$ is that it gives equal weight to all variants, regardless of their frequencies in the population.
Thus as a measure of diversity we focused on the reciprocal of Simpson's index $D$ \cite{Hill1973}, which do obey the replication principle.
The replication principle states that if we have N equally large, equally diverse groups of PRDM9 alleles with no alleles in common, the diversity of the pooled groups must be N times the diversity of a single group.
$\Dt$ equals $K_{t}$ if all the variants have equal weights and is close to $1$ if an allele has appreciable abundance.

\begin{align}
     \D &= \avg{\Dt} = \avg{\dfrac{1}{\sum_{i \in K_{t}} x_{i,t}^2}} 
\end{align}

2. Mean relative recombination rate

$\Rp _{i,t}$ is the relative recombination rate of the $i^{th}$ PRDM9 allele. Thus at the population level, the recombination rate is obtained by taking the weighted mean.

\begin{align}
    \R &= \avg{\Rt} = \avg{\sum_{i \in K_{t}} x_{i,t} \Rp _{i,t}} 
\end{align}

3. Landscape of the hotspots

Considering the number of hotspots in the genome is small compare to the 

\begin{align}
     \V &= \avg{\Vt} = \avg{\sum_{i \in K_{t}} x_{i,t}^2 \Rp _{i,t}} 
\end{align}

4. Turn-over of PRDM9

The turn-over is defined as the decorrelation time of the relative cross-homozygosity.
It is such that 

\begin{align}
    \avg{\dfrac{\sum_{i \in K_{t}} x_{i,t} x_{i,t+T}}{\sum_{i \in K_{t}} x_{i,t} x_{i,t}}  }  =  \dfrac{1}{2}
\end{align}

\section*{Results}
\subsection*{Simulation results} 

Brief reminder about model. Wright-Fisher with mutation and selection. Only PRMD9 locus; evolutionary dynamics at PRDM9 target loci (hotspots) is implicit.

Figure scaling relations !

Equilibrium recombination rate is suboptimal ($\R < 1)$. Equivalently there is a recombination load associated to the red-queen.

Mean equilibrium recombination rate ($\R$) is constant as a function of $\Ne$, increases with $u$ and decreases with $v$.

Plotting the PRDM9 diversity ($\D$) as a function of $\Ne$ and $u$ reveals two distinct regimes: succession regime and polymorphic regime, essentially depending on the scaled mutation rate ($\Ne u$) at the PRDM9 locus. For low scaled mutation rates, one PRDM9 alleles at a time. For high scaled mutation rate, multiple alleles segregate in the population: in the regime $\D$ is roughly proportional to $\Ne u$.

In succession regime: turn-over time ($\T$) decreases with $\Ne$, reaching a maximum at the transition between the two regimes. Then, turn-over time remains constant.

Conversely, variance of recombination landscape ($\V$) first remains constant as a function of $\Ne$ in succession regime, then starts to decrease with $\Ne$ in polymorphic regime.

Thus as a function of $\Ne$, red-queen first unfolds as a succession of PRDM9 alleles, one at a time: recombination is then always concentrated on one single set of hotspots (corresponding to current dominant PRDM9 allele). As $\Ne$ increases, red-queen accelerates, up to a point where successive waves of erosion-invasion start to overlap, such that multiple PRDM9 alleles start to coexist in the population. In this regime, turn-over time now remains constant, as a function of $\Ne$. On the other hand, diversity at the PRDM9 locus increases, such that recombination is progressively spread over an increasing number of weaker hot-spots.

\subsection*{Analytical approximations in succession regime} 

In a succession regime, PRDM9 alleles don't co-segregate. There is only one allele at a time, and it's recombination rate ( $\Rp_{t}$) is decreasing proportionally to itself (biased gene conversion) at a rate $\rho = \Ne v r_0$. Thus $\Rp_{t}$ follow an exponential decrease.

\begin{align} 
   \dfrac{\dd \Rp_{t}}{\dd t}  &=  - \rho \Rp_{t} \\
        \Rightarrow \Rp_{t} &= \operatorname{e}^{-\rho t}
\end{align}

The recombination rate  ($\Rp_{t}$) decreases up to point where a new PRDM9 allele invade the population. We call $\tau$ the mean time between two invasion of alleles. Thus when new alleles are invading, the resident alleles recombination rate are $\Rmin$ and we have :
\begin{equation}
   \Rmin =  \operatorname{e}^{-\rho \tau}
\end{equation}

Moreover the mean recombination rate $\R$ of alleles between there invasion and extinctions can be derived explicitly and leads to a relation between $\R$ and $\Rmin$: 

\begin{align} 
	R &= \avg{\Rp_{t}} \\
	\Rightarrow 0 &=  1-\Rmin  + \R  \operatorname{log}(\Rmin ) 
\end{align}

Using population genetic arguments, we can derive the substitution rate between $\mu_{t}$ that is the probability of fixation of a new allele that mutates at rate $u$ in a population of size $\Ne$ with selection coefficient $ s(\Rp_{t}) $. The selection coefficient is derived in appendix after linearisation and leads to the equation : 

\begin{equation}
\mu_{t} = 2 \Ne u s(\Rp_{t}) = \Ne u \dfrac{f'(\Rp_{t})}{f(\Rp_{t} )} (1 - \Rp_{t} ) 
\end{equation}

And the mean time between two invasion of alleles ($\tau$) can be also derived as $\avg{\mu_{t}^{-1}}$  .

\begin{equation}
\tau = \avg{\mu_{t}^{-1}} = \dfrac{1}{ \Ne u}\avg{\dfrac{f(\Rp_{t} )}{f'(\Rp_{t})(1 - \Rp_{t} )}}  = \dfrac{f(\R )}{ \Ne u f'(\R)(1 - \R) }
\end{equation}

And thus we get a self-consistent equation on $\R$ 
\begin{align} 
	& \dfrac{f'(\R )}{f(\R )} \dfrac{(1- \R )(1- \Rmin )}{\R } = \dfrac{v r_o}{ u} \\
	\Rightarrow  & \R= g\left(\dfrac{v r_o}{ u}\right)
\end{align}

Where $g$ is a strictly decreasing function cannot be solved explicitly but can be approximated (see section small load approximation). 
It is important to note that $\R$ is the independent if the population size $\Ne$, as our simulation was suggesting.

In the succession regime, the diversity at the PRDM9 locus $\D$ is simply equal $1$ since there is only one allele.

In the succession regime, the landscape of hotspots is $\V = \R $ ?????

And finally, we get the turn-over time $\T$ 
\begin{equation}
  \T =  \tau = \dfrac{1 - \Rmin }{\rho \R }
\end{equation}


\subsection*{Analytical approximations in polymorphic regime}

By discarding the drift and linearising the activity, we derived a close set of differential equations for the frequencies of PRDM9 alleles ($x_{i,t}$) and there associated recombination rate ($\Rp_{i,t}$). 

\begin{equation}
\left\{
      \begin{aligned}
          \dfrac{\dd x_{i,t}}{\dd t} &= \dfrac{f'(\Rt )}{2 f(\Rt )} \left( \Rp_{i,t}  - \Rt  \right) x_{i,t}, \;
             \forall i \in \{ 1, \, \dots, \, K_{t} \} \\
        \dfrac{\dd \Rp_{i,t}}{\dd t} &= 
        - \rho x_{i,t} \Rp_{i,t}, \;
             \forall i \in \{ 1, \, \dots, \, K_{t} \} \\
             \Rt &= \sum_{i \in K_{t}} x_{i,t} \Rp _{i,t} 
      \end{aligned}
\right. 
\end{equation}

Under the assumption of many alleles co-segregating in the population and under a mean field approximation, then $\Rt=\R$ is independent of the trajectory followed by a single allele. This leads to a decoupling of the above system of equation, and we can study the trajectory of a single allele with frequency $x_{t}$ and relative recombination rate $\Rp _{t}$.
\begin{equation}
\left\{
      \begin{aligned}
          \dfrac{\dd x_{t}}{\dd t} &= \dfrac{f'(\R )}{2 f(\R )} \left( \Rp_{t}  - \R \right) x_{t} \\
        \dfrac{\dd \Rp_{t}}{\dd t} &= 
        - \rho x_{t} \Rp_{t} \\
      \end{aligned}
\right.
\end{equation}
Show plot and describe overall structure.

System of equations is not analytically solvable as a function of $t$. On the other hand, $\Rp_{t}$ is monotonic, and $x$ can be analytically expressed as a function of $\Rp_{t}$
\begin{equation}
x(\Rp_{t} ) =\dfrac{f'(\R)}{2 \rho f(\R )} (1- \Rp_{t}  + \R  \operatorname{log}(\Rp_{t} )) + x_{\mathrm{initial}}
\end{equation}

Since $x_{\mathrm{initial}} = 1 / \Ne \simeq 0$, we get an equation on $\R$ and $\Rmin$ when $x$ goes to zero.

\begin{align}
 	0 &=  x(\Rmin) \\
   \Rightarrow 0 &\simeq  1-\Rmin  + \R  \operatorname{log}(\Rmin ) 
\end{align}

Surprisingly, this is the same equation as in succession regime.

So far we considered $\R$ as an external parameter for a focal allele, but if every allele has the same trajectory, but with different arrival time, they all contribute back to $\R$. Thus we get a self-consistent estimation for  $\R$ using a tilling argument (see figure x). 

\begin{align} 
  & \left\{
  \begin{aligned}
    \tau &= \int_{0}^{\infty } x_{t} \dd  t = \dfrac{1 - \Rmin }{\rho \R } \\
    \tau &= \left( 2 \Ne u \dfrac{f'(\R )}{2f(\R )} (1-\R )  \right)^{-1}
  \end{aligned}
  \right. \\
   \Rightarrow &
    \dfrac{f'(\R )}{f(\R )} \dfrac{(1- \R )(1- \Rmin )}{\R } = \dfrac{v r_o}{ u}
\end{align}
Again surprisingly, we get the same self-consistent equation as in succession regime.

Using the same tilling argument, we can get the diversity at the PRDM9 locus $\D$
\begin{equation}
  \D = \dfrac{\int_{0}^{\infty } x_{t}^2 \dd  t}{\tau} = \dfrac{4\rho f(\R )}{f'(\R )\left[ 1 + \Rmin  - 2 \R   \right]} 
\end{equation}
This equation is not analytically solvable as a function of the parameters, we must first numerically solve the self-consistent of $\R$ and compute $\Rmin$ to estimate $\D$.

In the same fashion, we get the landscape of hotspots $\V$
\begin{equation}
    \V = \dfrac{\int_{0}^{\infty } x_{t}^2 \Rp_{t} \dd  t}{\tau} = \dfrac{f'(\R )}{4 \rho f(\R )}\R \left[ 1 + \Rmin  - 2 \R   \right] 
\end{equation}

And finally, we get the turn-over time $\T$ 
\begin{equation}
  \T = \dfrac{\D}{\Ne u \dfrac{f'(\R )}{2 f(\R )} (1-\R )}  =  \left( \dfrac{ f(\R )}{f'(\R )} \right)^2  \dfrac{1}{(1-\R )( 1 + \Rmin  - 2 \R   )}  \dfrac{8 v r_0}{u}
\end{equation}

\subsection*{Small load development}
The first approximation is to set a power law function for the fitness; $f(x)=x^{\alpha}$, thus $f'(\R)/f(\R)= \alpha / \R$.

Let us define $\epsilon = \sqrt{\dfrac{v r_0}{\alpha u}}$

\begin{equation}
   \dfrac{(1- \R )(1- \Rmin )}{\R^2} =  \epsilon^2
\end{equation}

\begin{align}
    \Rmin  &=\frac{7}{4}-\frac{\sqrt{48-39  \R}}{4 \sqrt{ \R}}
\end{align}

\begin{align} 
    & \dfrac{(1- \R )(1- \Rmin )}{\R^2} =  \epsilon \\
    & \dfrac{(1- \R )\left(\dfrac{\sqrt{48-39 \R}}{4 \sqrt{\R}} -\frac{3}{4}\right)}{\R^2} =  \epsilon \\    
    \Rightarrow &
    \left\{
  \begin{aligned}
     \R =  1 - \dfrac{1}{\sqrt{2}}\sqrt{\epsilon} + \dfrac{5}{12} \epsilon + \circ ( \epsilon)\\
     \Rmin =  1 - \sqrt{2} \sqrt{\epsilon} + \dfrac{7}{6} \epsilon + \circ ( \epsilon)\\
     \end{aligned}
  \right.
\end{align}

with $\dfrac{v r_o}{\alpha u} = \epsilon$


1. Approximation of $D$

\begin{align}
  \D &= \dfrac{4\rho \alpha }{\R \left[ 1 + \Rmin  - 2 \R   \right]} \\
   & \simeq \dfrac{12 \rho}{\epsilon} = 12 N_e u
\end{align}

2. Estimation of $V$

\begin{align}
  \V &= \dfrac{\R }{4 \rho }\left[ 1 + \Rmin  - 2 \R   \right] \\
   &\simeq \dfrac{\epsilon}{12 \rho} = \dfrac{1}{12 N_e u}
\end{align}

3. Estimation of $ T $

\begin{align}
   \T &= \D \left( \Ne u \alpha \dfrac{1-\R }{\R }   \right)^{-1} \\
    &\simeq 12 \sqrt{\dfrac{2 }{\alpha \epsilon }} =  12 \sqrt{\dfrac{2 u }{\alpha v r_0}}
\end{align}

\section*{Discussion}

\subsection*{Limits of the models}

- It's a toy model, not a predictive model, mainly it gives us insight on the red dynamic.

- Implicit substitution rate at the hotspots, we should investigate a full model with population genetics on both arms of the red queen.

- Exponential decrease of recombination rate: if PRDM9 is limiting in the cell for recombination (not the hotspots) this doesn't hold true.

- The time between invasion must be longer than the time for an allele to invade the population in succession. 

- Structure of populations are not taken in account, this is a major shortcoming since PRDM9 is known to be 

- Fitness function shape is important (cf appendix)

- Small load development approximation

- Transition between succession and polymorphic regime is not fully resolved.



\subsection*{Parameters of the model versus empirical data}
- Testing against empirical data currently available in mammals: 
1. Population size $\Ne$

2. Mutation rate at the PRDM9 locus $u$

3. Erosion rate at the hotspots $v$

4. Recombination rate at the hotspots $r_0$

5. Fitness function $f$

\subsection*{Mathematical results against empirical data}
- Testing against empirical data currently available in mammals. $12 \Ne u \gg 1$ in mouse, and we observe $\D \gg 1$. In humans we have $12 \Ne u \simeq 1$, and we observe $\D \simeq 1$.
- Our scaling relations can now be tested 

\bibliographystyle{plain}
\bibliography{red-queen}

\part*{Appendix}
\section*{Approximation for the fitness function}
Let us denote $\R_{t} =\sum_{i  \in K_{t}} x_{i,t} \Rp _{i,t}$.

One can use a Taylor approximation to linearise the fitness function around $\R $.  

\begin{align}
    \overline{\omega_{i,t}} - f(\R_{t} ) &=
    \sum_{j \in K_{t}} x_{j,t} f \left( \dfrac{\Rp _{i,t} + \Rp _{j,t}}{2} \right) - \sum_{j \in K_{t}} x_{j,t} f(\R_{t} ) \\
    &=
    \sum_{j \in K_{t}} x_{j,t}  \left[ f \left( \dfrac{\Rp _{i,t} + \Rp _{j,t}}{2} \right) - f(\R_{t} ) \right] \\
    &\simeq
    \sum_{j \in K_{t}} x_{j,t}  f'(\R_{t} ) \left( \dfrac{\Rp _{i,t} + \Rp _{j,t}}{2} - \R_{t}  \right) \\
    &\simeq
     f'(\R_{t} ) \left( \dfrac{\Rp _{i,t} + \sum_{j \in K_{t}} x_{j,t} \Rp _{j,t}}{2} - \R_{t}  \right) \\
     &\simeq
     f'(\R_{t} ) \left( \dfrac{\Rp _{i,t} - \R_{t} }{2}\right) \\
\end{align}

Consequently $\overline{\omega_{t}} = f(\R_{t} )$. \textbf{Proof},

\begin{align}
    f(\R_{t} ) &= \sum_{i \in K_{t}} x_{i,t} f(\R_{t} ) \\
    &\simeq \sum_{i \in K_{t}} x_{i,t} \left[ \overline{\omega_{i,t}} - f'(\R_{t} ) \left( \dfrac{\Rp _{i,t} - \R_{t} }{2}\right) \right] \\
    &\simeq
    \sum_{i \in K_{t}} x_{i,t} \overline{\omega_{i,t}} - f'(\R_{t} ) \sum_{i \in K_{t}} x_{i,t}  \left( \dfrac{\Rp _{i,t} - \R_{t} }{2}\right) \\
    &\simeq
     \overline{\omega_{t}} - f'(\R_{t} )  \left( \dfrac{\sum_{i \in K_{t}} x_{i,t} \Rp _{i,t} - \R_{t} }{2}\right) \\
    &\simeq
     \overline{\omega_{t}}
\end{align}

And the selection coefficient $s_{i,t}$ for the variant $i$ can be easily computed

\begin{equation}
    s_{i,t} = \dfrac{\overline{\omega_{i,t}} - \overline{\omega_{t}}}{\overline{\omega_{t}}}
    \simeq  \dfrac{f'(\R_{t} )}{f(\R_{t} )} \left[ \dfrac{\Rp _{i,t} - \R_{t} }{2} \right]
\end{equation}

Thus the probability of drawing variant $i$ in the multinomial distribution is given by 
\begin{equation}
    x_{i,t} \dfrac{\overline{\omega_{i,t}}}{\overline{\omega_{t}}} = 
     x_{i,t} (1 + s_{i,t}) =  x_{i,t} \left(1 +  \dfrac{f'(\R_{t} )}{f(\R_{t} )} \left[ \dfrac{\Rp _{i,t} - \R_{t} }{2} \right] \right)
\end{equation}

\section*{Mean recombination rate in succession regime}

\begin{align} 
    \R &=  \dfrac{1}{\tau}\int_0^{\tau} \Rp_{t} 
        	 = \dfrac{1}{\tau}\int_0^{\tau} \operatorname{e}^{-\rho t} \\
        	 &= \dfrac{1}{\tau} \left[  \dfrac{\operatorname{e}^{-\rho t}}{\rho }  \right]_{\tau}^{0}
        	 = \dfrac{\Rmin - 1}{\tau \rho}\\
        	 &= \dfrac{\Rmin - 1}{\operatorname{log}(\Rmin )} \\
    \Rightarrow 0 &=  1-\Rmin  + \R  \operatorname{log}(\Rmin ) 
\end{align}

\section*{Frequency as function of recombination rate in polymorphic regime.}

\begin{align} 
& \left\{
      \begin{aligned}
          \dfrac{\dd x_{t}}{\dd t} &= \dfrac{f'(\R )}{2 f(\R )} \left( \Rp_{t}  - \R \right) x_{t} \\
        \dfrac{\dd \Rp_{t}}{\dd t} &= 
        - \rho x_{t} \Rp_{t} \\
      \end{aligned}
\right. \\
 \Rightarrow
 & \left\{
      \begin{aligned}
          \dfrac{\dd x_{t}}{\dd \Rp_{t} } &= \dfrac{f'(\R )}{2 \rho f(\R )}\left( \dfrac{\R }{\Rp_{t} } -1 \right) \\
        \dfrac{\dd \Rp_{t} }{\dd t} &= 
        - \rho x_{t} \Rp_{t} \\
      \end{aligned}
    \right. \\
 \Rightarrow
  & \left\{
      \begin{aligned}
         x(\Rp_{t} ) &=\dfrac{f'(\R)}{2 \rho f(\R )} (1- \Rp_{t}  + \R  \operatorname{log}(\Rp_{t} )) + x_{\mathrm{initial}}  \\
        \dfrac{\dd \Rp_{t} }{\dd t} &= 
         \dfrac{f'(\R )}{2 f(\R )} [ \Rp_{t} -1- \R  \operatorname{log}(\Rp_{t} )]\Rp_{t}   - \rho x_{\mathrm{initial}} \Rp_{t} \\
      \end{aligned}
    \right.
\end{align}

\section*{$\tau$, $\D$ and $\V$ in polymorphic regime.}
We make use of the three equations :  
\begin{equation}
\left\{
  \begin{aligned}
         x(\Rp_{t} ) &= \dfrac{f'(\R)}{2 \rho f(\R )} (1- \Rp_{t}  + \R  \operatorname{log}(\Rp_{t} )) \\
         \dfrac{\dd t }{\dd \Rp_{t}} &= \dfrac{1}{-\rho x_{t} \Rp_{t}} \\
        0 &= 1-\Rmin  + \R  \operatorname{log}(\Rmin )  \\
  \end{aligned}
   \right.
\end{equation}

\begin{align}
\tau &= \int_{0}^{\infty } x_{t} \dd t  \\
    &= \int_{1}^{\Rmin } x(\Rp_{t} )   \dfrac{\dd t }{\dd  \Rp_{t}} \dd \Rp_{t}    \\
    &= \int_{1}^{\Rmin } x(\Rp_{t} )   \dfrac{1 }{- \rho x(\Rp_{t} ) \Rp_{t} } \dd \Rp_{t}    \\
    &=  \dfrac{-1 }{\rho }  \int_{1}^{\Rmin }   \dfrac{1 }{\Rp_{t} } \dd \Rp_{t}    \\
    &=  \dfrac{-1 }{\rho } \left[ \operatorname{log}(\Rp_{t}) \right]_{1}^{\Rmin }    \\
    &=  \dfrac{-1 }{\rho } \operatorname{log}(\Rmin )    \\
    &=  \dfrac{1 - \Rmin }{\rho \R } \displaybreak[3] \\ 
\D &= \left( \dfrac{\int_{0}^{\infty } x_{t}^2 \dd  t}{ \tau } \right)^{-1} \\
    &=   \tau \left( \int_{0}^{\infty } x_{t}^2 \dd  t \right)^{-1} \\
    &=   \tau  \left(\int_{1}^{\Rmin } x(\Rp_{t} )^2 \dfrac{\dd t }{\dd  \Rp_{t}} \dd \Rp_{t} \right)^{-1} \\
    &=   \dfrac{ \operatorname{log}(\Rmin ) }{-\rho  }  \left( \int_{1}^{\Rmin } x(\Rp_{t} )^2 \dfrac{1 }{- \rho x(\Rp_{t} ) \Rp_{t} } \dd \Rp_{t} \right)^{-1} \\
    &=   \operatorname{log}(\Rmin ) \left( \int_{1}^{\Rmin } \dfrac{x(\Rp_{t} ) }{ \Rp_{t} } \dd \Rp_{t} \right)^{-1} \\ 
    &=  \operatorname{log}(\Rmin ) \left( \int_{1}^{\Rmin } \dfrac{f'(\R)}{2 \rho f(\R )} \dfrac{ 1- \Rp_{t}  + \R  \operatorname{log}(\Rp_{t} ) }{ \Rp_{t} } \dd \Rp_{t} \right)^{-1} \\ 
    &=  \dfrac{2 \rho f(\R )}{f'(\R)} \operatorname{log}(\Rmin ) \left( \left[  - \Rp_{t} + \operatorname{log}(\Rp_{t}) + \dfrac{\R \operatorname{log}(\Rp_{t})^2}{2}  \right]_{1}^{\Rmin } \right)^{-1} \\ 
	&=  \dfrac{2 \rho f(\R )}{f'(\R)} \operatorname{log}(\Rmin ) \left( 1 - \Rmin + \operatorname{log}(\Rmin) + \dfrac{\R \operatorname{log}(\Rmin)^2}{2} \right)^{-1} \\ 
	&=  \dfrac{2 \rho f(\R )}{f'(\R)} \operatorname{log}(\Rmin ) \left( -\R \operatorname{log}(\Rmin) + \operatorname{log}(\Rmin) + \dfrac{\R \operatorname{log}(\Rmin)^2}{2} \right)^{-1} \\ 
	&=  \dfrac{2 \rho f(\R )}{f'(\R)}  \left(   -\R + 1 + \dfrac{\R \operatorname{log}(\Rmin)}{2}  \right)^{-1} \\ 
	&=  \dfrac{2 \rho f(\R )}{f'(\R)}  \left(  \dfrac{ -2\R + 2 + \Rmin - 1 }{2}  \right)^{-1} \\ 
    &= \dfrac{4 \rho f(\R )}{f'(\R )\left[ 1 + \Rmin  - 2 \R   \right]} \displaybreak[3] \\ 
\V &= \dfrac{\int_{0}^{\infty } x_{t}^2  \Rp_{t} \dd  t}{ \tau }  \\
    &=   \dfrac{1}{\tau}  \left(\int_{1}^{\Rmin } x(\Rp_{t} )^2 \Rp_{t} \dfrac{\dd t }{\dd  \Rp_{t}} \dd \Rp_{t} \right)^{-1} \\
    &=  \dfrac{ \rho \R }{1 - \Rmin }  \int_{1}^{\Rmin } x(\Rp_{t} )^2 \Rp_{t} \dfrac{1 }{- \rho x(\Rp_{t} ) \Rp_{t} } \dd \Rp_{t}  \\
    &=   \dfrac{ \R }{\Rmin - 1} \int_{1}^{\Rmin } x(\Rp_{t} )  \dd \Rp_{t} \\ 
    &=   \dfrac{ \R }{\Rmin - 1} \int_{1}^{\Rmin } \dfrac{f'(\R)}{2 \rho f(\R )} [ 1- \Rp_{t}  + \R  \operatorname{log}(\Rp_{t} )  ] \dd \Rp_{t} \\ 
    &=  \dfrac{f'(\R)}{2 \rho f(\R )} \dfrac{ \R }{\Rmin - 1} \left[  \Rp_{t} - \dfrac{\Rp_{t}^2}{2} - \Rp_{t} \R + \Rp_{t} \R \operatorname{log}(\Rp_{t} )   \right]_{1}^{\Rmin }  \\ 
	&=  \dfrac{f'(\R)}{2 \rho f(\R )} \dfrac{ \R }{\Rmin - 1} \left(  R - \dfrac{1}{2} + \Rmin - \dfrac{\Rmin^2}{2} - \Rmin \R + \Rmin \R \operatorname{log}(\Rmin )   \right)  \\
	&=  \dfrac{f'(\R)}{2 \rho f(\R )} \dfrac{ \R }{\Rmin - 1} \dfrac{ 2 R - 1 + 2\Rmin - \Rmin^2 - 2 \Rmin \R + 2 \Rmin (\Rmin - 1 )  }{2}  \\
	&=  \dfrac{f'(\R)}{2 \rho f(\R )} \dfrac{ \R }{\Rmin - 1} \left( 2 R - \Rmin - 1  + \Rmin + \Rmin^2 - 2 \Rmin \R \right)  \\
	&=  \dfrac{f'(\R)}{2 \rho f(\R )} \dfrac{ \R }{\Rmin - 1} (\Rmin - 1 )\left(  1 + \Rmin  - 2 \R \right)  \\
	&= \dfrac{f'(\R )}{4 \rho f(\R )}\R \left[ 1 + \Rmin  - 2 \R   \right] 
\end{align}

\section*{Small load development}
Let us define $\Lr = 1 - \R $ is the recombination load and $\Lmin = 1 - \Rmin $ is the maximum recombination load.
We make as second order development on $\epsilon$ for both $\Lr$ and $\Lmin$
\begin{align}
&\left\{
  \begin{aligned}
         \Lr &= \Lra \epsilon + \Lrb \epsilon^2 + \circ ( \epsilon^2) \\
         \Lmin &= \Lmina \epsilon + \Lminb \epsilon^2 + \circ ( \epsilon^2)  \\
  \end{aligned}
   \right.
   \\ \Rightarrow
&\left\{
  \begin{aligned}
         \Lr^2 &= \Lra^2 \epsilon^2 + \circ ( \epsilon^2) \\
         \Lmin^2 &= \Lmina^2 \epsilon^2 + \circ ( \epsilon^2)  \\
         \Lmin \Lr&= \Lmina \Lra \epsilon^2 + (\Lmina \Lrb + \Lminb \Lra)  \epsilon^3 + \circ ( \epsilon^3)   \\
  \end{aligned}
   \right.
\end{align}

\begin{align}
&\left\{
  \begin{aligned}
         0 &= 1-\Rmin  + \R  \operatorname{log}(\Rmin ) \\
         \epsilon^2 &=  \dfrac{(1- \R )(1- \Rmin )}{\R^2} 
  \end{aligned}
   \right.
   \\ \Rightarrow
&\left\{
  \begin{aligned}
         0 &= \Lmin  + ( 1- \Lr) \operatorname{log}(1 - \Lmin ) \\
         \epsilon^2 &=  \dfrac{ \Lmin \Lr }{(1- \Lr)^2} 
  \end{aligned}
   \right.
    \\ \Rightarrow
   &\left\{
  \begin{aligned}
       &  -\Lmin = ( 1- \Lr) \left(-\Lmin - \dfrac{\Lmin^2}{2} - \dfrac{\Lmin^3}{3} + \circ ( \Lmin^3) \right) \\
       &   \Lmin \Lr =   (1- \Lr)^2 \epsilon^2
  \end{aligned}
   \right.
       \\ \Rightarrow
   &\left\{
  \begin{aligned}
       &  1 = ( 1- \Lr) \left(1 + \dfrac{\Lmin}{2} + \dfrac{\Lmin^2}{3} + \circ ( \Lmin^2) \right) \\
       &   \Lmin \Lr =   (1- \Lr)^2 \epsilon^2
  \end{aligned}
   \right.
  \\ \Rightarrow
   &\left\{
  \begin{aligned}
       &  1 = [ 1- \Lra \epsilon - \Lrb \epsilon^2 + \circ ( \epsilon^2) ] \left(1 + \dfrac{\Lmina \epsilon }{2} + \dfrac{ \Lminb \epsilon^2}{2}  + \dfrac{\Lmina^2 \epsilon^2}{3} + \circ ( \epsilon^2) \right) \\
       &   \Lmina \Lra \epsilon^2 + (\Lmina \Lrb + \Lminb \Lra)  \epsilon^3 + \circ ( \epsilon^3) =   (1- \Lra \epsilon - \Lrb \epsilon^2 + \circ ( \epsilon^2))^2 \epsilon^2
  \end{aligned}
   \right.
     \\ \Rightarrow
   &\left\{
  \begin{aligned}
       &  1 = 1 + \dfrac{\Lmina - 2 \Lra }{2} \epsilon + \dfrac{ 2 \Lmina^2 - 3 \Lmina \Lra + 3 \Lminb - 6 \Lrb  }{6} \epsilon^2 + \circ ( \epsilon^2)  \\
       &   \Lmina \Lra + (\Lmina \Lrb + \Lminb \Lra)  \epsilon + \circ ( \epsilon) =   1 - 2 \Lra \epsilon + \circ ( \epsilon )
  \end{aligned}
   \right.
  \\ \Rightarrow
   &\left\{
     \begin{aligned}
       &  0 = \Lmina - 2 \Lra \\
       &  0 = 2 \Lmina^2 - 3 \Lmina \Lra + 3 \Lminb - 6 \Lrb \\
       &  0 = 1 - \Lmina \Lra   \\
       &  0 = \Lmina \Lrb + \Lminb \Lra + 2 \Lra \\
  \end{aligned}
   \right.
     \\ \Rightarrow
   &\left\{
   \begin{aligned}
       &  \Lmina = \sqrt{2} \\
       &  \Lminb= \dfrac{1}{\sqrt{2}} \\
       &  \Lra = \dfrac{-7}{6}  \\
       &  \Lrb = \dfrac{-5}{12} \\
  \end{aligned}
     \right.
       \\ \Rightarrow
   &\left\{
     \begin{aligned}
     \R  &=  1 - \dfrac{1}{\sqrt{2}}\epsilon + \dfrac{5}{12} \epsilon^2 + \circ ( \epsilon^2)\\
     \Rmin & =  1 - \sqrt{2} \epsilon + \dfrac{7}{6} \epsilon^2 + \circ ( \epsilon^2)\\
  \end{aligned}
   \right.
\end{align}

\end{document}
